% Options for packages loaded elsewhere
\PassOptionsToPackage{unicode}{hyperref}
\PassOptionsToPackage{hyphens}{url}
\PassOptionsToPackage{dvipsnames,svgnames,x11names}{xcolor}
%
\documentclass[
  letterpaper,
  DIV=11,
  numbers=noendperiod]{scrartcl}

\usepackage{amsmath,amssymb}
\usepackage{iftex}
\ifPDFTeX
  \usepackage[T1]{fontenc}
  \usepackage[utf8]{inputenc}
  \usepackage{textcomp} % provide euro and other symbols
\else % if luatex or xetex
  \usepackage{unicode-math}
  \defaultfontfeatures{Scale=MatchLowercase}
  \defaultfontfeatures[\rmfamily]{Ligatures=TeX,Scale=1}
\fi
\usepackage{lmodern}
\ifPDFTeX\else  
    % xetex/luatex font selection
\fi
% Use upquote if available, for straight quotes in verbatim environments
\IfFileExists{upquote.sty}{\usepackage{upquote}}{}
\IfFileExists{microtype.sty}{% use microtype if available
  \usepackage[]{microtype}
  \UseMicrotypeSet[protrusion]{basicmath} % disable protrusion for tt fonts
}{}
\makeatletter
\@ifundefined{KOMAClassName}{% if non-KOMA class
  \IfFileExists{parskip.sty}{%
    \usepackage{parskip}
  }{% else
    \setlength{\parindent}{0pt}
    \setlength{\parskip}{6pt plus 2pt minus 1pt}}
}{% if KOMA class
  \KOMAoptions{parskip=half}}
\makeatother
\usepackage{xcolor}
\setlength{\emergencystretch}{3em} % prevent overfull lines
\setcounter{secnumdepth}{-\maxdimen} % remove section numbering
% Make \paragraph and \subparagraph free-standing
\ifx\paragraph\undefined\else
  \let\oldparagraph\paragraph
  \renewcommand{\paragraph}[1]{\oldparagraph{#1}\mbox{}}
\fi
\ifx\subparagraph\undefined\else
  \let\oldsubparagraph\subparagraph
  \renewcommand{\subparagraph}[1]{\oldsubparagraph{#1}\mbox{}}
\fi


\providecommand{\tightlist}{%
  \setlength{\itemsep}{0pt}\setlength{\parskip}{0pt}}\usepackage{longtable,booktabs,array}
\usepackage{calc} % for calculating minipage widths
% Correct order of tables after \paragraph or \subparagraph
\usepackage{etoolbox}
\makeatletter
\patchcmd\longtable{\par}{\if@noskipsec\mbox{}\fi\par}{}{}
\makeatother
% Allow footnotes in longtable head/foot
\IfFileExists{footnotehyper.sty}{\usepackage{footnotehyper}}{\usepackage{footnote}}
\makesavenoteenv{longtable}
\usepackage{graphicx}
\makeatletter
\def\maxwidth{\ifdim\Gin@nat@width>\linewidth\linewidth\else\Gin@nat@width\fi}
\def\maxheight{\ifdim\Gin@nat@height>\textheight\textheight\else\Gin@nat@height\fi}
\makeatother
% Scale images if necessary, so that they will not overflow the page
% margins by default, and it is still possible to overwrite the defaults
% using explicit options in \includegraphics[width, height, ...]{}
\setkeys{Gin}{width=\maxwidth,height=\maxheight,keepaspectratio}
% Set default figure placement to htbp
\makeatletter
\def\fps@figure{htbp}
\makeatother

\usepackage{booktabs}
\usepackage{longtable}
\usepackage{array}
\usepackage{multirow}
\usepackage{wrapfig}
\usepackage{float}
\usepackage{colortbl}
\usepackage{pdflscape}
\usepackage{tabu}
\usepackage{threeparttable}
\usepackage{threeparttablex}
\usepackage[normalem]{ulem}
\usepackage{makecell}
\usepackage{xcolor}
\usepackage[auth-lg]{authblk}
\KOMAoption{captions}{tableheading}
\makeatletter
\makeatother
\makeatletter
\makeatother
\makeatletter
\@ifpackageloaded{caption}{}{\usepackage{caption}}
\AtBeginDocument{%
\ifdefined\contentsname
  \renewcommand*\contentsname{Table of contents}
\else
  \newcommand\contentsname{Table of contents}
\fi
\ifdefined\listfigurename
  \renewcommand*\listfigurename{List of Figures}
\else
  \newcommand\listfigurename{List of Figures}
\fi
\ifdefined\listtablename
  \renewcommand*\listtablename{List of Tables}
\else
  \newcommand\listtablename{List of Tables}
\fi
\ifdefined\figurename
  \renewcommand*\figurename{Figure}
\else
  \newcommand\figurename{Figure}
\fi
\ifdefined\tablename
  \renewcommand*\tablename{Table}
\else
  \newcommand\tablename{Table}
\fi
}
\@ifpackageloaded{float}{}{\usepackage{float}}
\floatstyle{ruled}
\@ifundefined{c@chapter}{\newfloat{codelisting}{h}{lop}}{\newfloat{codelisting}{h}{lop}[chapter]}
\floatname{codelisting}{Listing}
\newcommand*\listoflistings{\listof{codelisting}{List of Listings}}
\makeatother
\makeatletter
\@ifpackageloaded{caption}{}{\usepackage{caption}}
\@ifpackageloaded{subcaption}{}{\usepackage{subcaption}}
\makeatother
\makeatletter
\@ifpackageloaded{tcolorbox}{}{\usepackage[skins,breakable]{tcolorbox}}
\makeatother
\makeatletter
\@ifundefined{shadecolor}{\definecolor{shadecolor}{rgb}{.97, .97, .97}}
\makeatother
\makeatletter
\makeatother
\makeatletter
\makeatother
\ifLuaTeX
  \usepackage{selnolig}  % disable illegal ligatures
\fi
\IfFileExists{bookmark.sty}{\usepackage{bookmark}}{\usepackage{hyperref}}
\IfFileExists{xurl.sty}{\usepackage{xurl}}{} % add URL line breaks if available
\urlstyle{same} % disable monospaced font for URLs
\hypersetup{
  pdftitle={Trabalho Prático 2},
  pdfauthor={Ana Carolina Vianna - 18/0097261; César Augusto Galvão - 19/0011572; Yan Flávio Vianna - 14/0166149},
  colorlinks=true,
  linkcolor={blue},
  filecolor={Maroon},
  citecolor={Blue},
  urlcolor={Blue},
  pdfcreator={LaTeX via pandoc}}

\title{Trabalho Prático 2}
\usepackage{etoolbox}
\makeatletter
\providecommand{\subtitle}[1]{% add subtitle to \maketitle
  \apptocmd{\@title}{\par {\large #1 \par}}{}{}
}
\makeatother
\subtitle{Análise de Séries Temporais - 1/2023}
\author{Ana Carolina Vianna - 18/0097261 \and César Augusto Galvão -
19/0011572 \and Yan Flávio Vianna - 14/0166149}
\date{}

\begin{document}
\maketitle
\ifdefined\Shaded\renewenvironment{Shaded}{\begin{tcolorbox}[sharp corners, enhanced, breakable, frame hidden, interior hidden, boxrule=0pt, borderline west={3pt}{0pt}{shadecolor}]}{\end{tcolorbox}}\fi

\renewcommand*\contentsname{Table of contents}
{
\hypersetup{linkcolor=}
\setcounter{tocdepth}{3}
\tableofcontents
}
\newpage{}

\hypertarget{introduuxe7uxe3o-suxe9rie-selecionada-caracteruxedsticas-e-decomposiuxe7uxe3o}{%
\section{Introdução: série selecionada, características e
decomposição}\label{introduuxe7uxe3o-suxe9rie-selecionada-caracteruxedsticas-e-decomposiuxe7uxe3o}}

A série temporal escolhida foi a de número \emph{id} correspondente a
2183. De acordo com a definição do próprio pacote, refere-se a
\emph{Fluid power shipments - hydraulic index}. Foram realizadas medidas
mensais de 1983 a 1992 e o horizonte de previsão requerido é das 18
ocorrências seguintes.

O gráfico da série, com \emph{in} e \emph{out-sample}, é exposto a
seguir.

\includegraphics{T2_grupo5_files/figure-pdf/plot-serie-total-1.pdf}

A série aparenta ter dois períodos, pelo menos: um ciclo anual e outro
que compreende um período maior. No entanto, ao se tentar decompor a
série com múltiplas sazonalidades, obté-se o seguinte:

\begin{itemize}
\tightlist
\item
  \textbf{Adicionando uma componente sazonal com ciclo menor que 1 ano}
  -- uma das componentes sazonais apresenta heteroscedasticidade;
\item
  \textbf{Adicionando uma componente sazonal com ciclo maior que 1 ano}
  -- resíduos apresentam periodicidade ou heteroscedasticidade.
\end{itemize}

Optou-se portanto pela decomposição STL (apesar de os dados terem
inicialmente formado um objeto \texttt{msts}) apenas com a sazonalidade
anual, mas fica evidente que esta decomposição não é adequada quando se
avalia a componente de tendência, que aparenta ainda carregar algum
componente periódico. Os resíduos aparentam um comportamento aleatório e
têm média -0.104, o que é próximo de zero o suficiente considerando a
magnitude dos dados da série. A decomposição é exposta a seguir.

\includegraphics{T2_grupo5_files/figure-pdf/grafico-mstl-1.pdf}

\hypertarget{modelos-arima-seleuxe7uxe3o-transformauxe7uxf5es-e-resuxedduos}{%
\section{Modelos ARIMA: seleção, transformações e
resíduos}\label{modelos-arima-seleuxe7uxe3o-transformauxe7uxf5es-e-resuxedduos}}

COMENTAR - para que serve diff normal e sazonal - o que conseguimos
depois de aplicar - teste de estacionariedade - deixar pronto para a
modelagem

\hypertarget{modelo-sem-transformauxe7uxe3o}{%
\subsection{Modelo sem
transformação}\label{modelo-sem-transformauxe7uxe3o}}

\hypertarget{seleuxe7uxe3o}{%
\subsubsection{Seleção}\label{seleuxe7uxe3o}}

\begin{verbatim}
[1] 1
\end{verbatim}

\begin{verbatim}
[1] 1
\end{verbatim}

\includegraphics{T2_grupo5_files/figure-pdf/acf-pacf-sem-transformacao-1.pdf}

\begin{verbatim}
 non-finite finite-difference value [1] 
\end{verbatim}

\begin{verbatim}
[1] 1380.949
\end{verbatim}

\begin{itemize}
\tightlist
\item
  Construção do modelo
\end{itemize}

O modelo selecionado foi ARIMA(2,1,2)x(0,1,2) para a série original e
com transformação de Box-Cox.

\hypertarget{resuxedduos}{%
\subsubsection{Resíduos}\label{resuxedduos}}

\hypertarget{modelo-com-transformauxe7uxe3o}{%
\subsection{Modelo com
transformação}\label{modelo-com-transformauxe7uxe3o}}

\hypertarget{seleuxe7uxe3o-1}{%
\subsubsection{Seleção}\label{seleuxe7uxe3o-1}}

\hypertarget{resuxedduos-1}{%
\subsubsection{Resíduos}\label{resuxedduos-1}}

\hypertarget{modelos-ets-seleuxe7uxe3o-transformauxe7uxf5es-e-resuxedduos}{%
\section{Modelos ETS: seleção, transformações e
resíduos}\label{modelos-ets-seleuxe7uxe3o-transformauxe7uxf5es-e-resuxedduos}}

\hypertarget{modelo-sem-transformauxe7uxe3o-1}{%
\subsection{Modelo sem
transformação}\label{modelo-sem-transformauxe7uxe3o-1}}

\hypertarget{seleuxe7uxe3o-2}{%
\subsubsection{Seleção}\label{seleuxe7uxe3o-2}}

\begin{longtable*}{lccc}
\toprule
Modelo & AIC & AICc & BIC\\
\midrule
\endfirsthead
\multicolumn{4}{@{}l}{\textit{(continued)}}\\
\toprule
Modelo & AIC & AICc & BIC\\
\midrule
\endhead

\endfoot
\bottomrule
\endlastfoot
\cellcolor{gray!15}{ETS(M,Ad,M)} & \cellcolor{gray!15}{1761.30} & \cellcolor{gray!15}{1768.36} & \cellcolor{gray!15}{1810.87}\\
ETS(M,M,M) & 1761.94 & 1769.00 & 1811.51\\
\cellcolor{gray!15}{ETS(Ad,A,A)} & \cellcolor{gray!15}{1764.25} & \cellcolor{gray!15}{1771.30} & \cellcolor{gray!15}{1813.81}\\
ETS(M,Ad,A) & 1767.73 & 1774.78 & 1817.29\\
\cellcolor{gray!15}{ETS(M,A,M)} & \cellcolor{gray!15}{1769.04} & \cellcolor{gray!15}{1775.29} & \cellcolor{gray!15}{1815.86}\\
ETS(A,A,A) & 1771.20 & 1777.44 & 1818.01\\
\cellcolor{gray!15}{ETS(M,A,A)} & \cellcolor{gray!15}{1774.56} & \cellcolor{gray!15}{1780.81} & \cellcolor{gray!15}{1821.38}\\
ETS(M,M,M) & 1781.06 & 1787.30 & 1827.87\\
\cellcolor{gray!15}{ETS(A,N,A)} & \cellcolor{gray!15}{1812.26} & \cellcolor{gray!15}{1817.06} & \cellcolor{gray!15}{1853.56}\\
ETS(M,N,M) & 1827.18 & 1831.98 & 1868.48\\*
\end{longtable*}

\includegraphics{T2_grupo5_files/figure-pdf/melhor-fit-ETL-sem-transf-1.pdf}

\hypertarget{resuxedduos-2}{%
\subsubsection{Resíduos}\label{resuxedduos-2}}

\hypertarget{modelo-com-transformauxe7uxe3o-1}{%
\subsection{Modelo com
transformação}\label{modelo-com-transformauxe7uxe3o-1}}

\hypertarget{seleuxe7uxe3o-3}{%
\subsubsection{Seleção}\label{seleuxe7uxe3o-3}}

a série com transformacao

\includegraphics{T2_grupo5_files/figure-pdf/ETS-com-transf-1.pdf}

decomposicao

\includegraphics{T2_grupo5_files/figure-pdf/decomposicao-ets-com-transformacao-1.pdf}

selecao do modelo com transformação

\begin{longtable*}{lccc}
\toprule
Modelo transformado & AIC & AICc & BIC\\
\midrule
\endfirsthead
\multicolumn{4}{@{}l}{\textit{(continued)}}\\
\toprule
Modelo transformado & AIC & AICc & BIC\\
\midrule
\endhead

\endfoot
\bottomrule
\endlastfoot
\cellcolor{gray!15}{ETS(M,Ad,M)} & \cellcolor{gray!15}{1761.30} & \cellcolor{gray!15}{1768.36} & \cellcolor{gray!15}{1810.87}\\
ETS(M,M,M) & 1761.94 & 1769.00 & 1811.51\\
\cellcolor{gray!15}{ETS(Ad,A,A)} & \cellcolor{gray!15}{1764.25} & \cellcolor{gray!15}{1771.30} & \cellcolor{gray!15}{1813.81}\\
ETS(M,Ad,A) & 1767.73 & 1774.78 & 1817.29\\
\cellcolor{gray!15}{ETS(M,A,M)} & \cellcolor{gray!15}{1769.04} & \cellcolor{gray!15}{1775.29} & \cellcolor{gray!15}{1815.86}\\
ETS(A,A,A) & 1771.20 & 1777.44 & 1818.01\\
\cellcolor{gray!15}{ETS(M,A,A)} & \cellcolor{gray!15}{1774.56} & \cellcolor{gray!15}{1780.81} & \cellcolor{gray!15}{1821.38}\\
ETS(M,M,M) & 1781.06 & 1787.30 & 1827.87\\
\cellcolor{gray!15}{ETS(A,N,A)} & \cellcolor{gray!15}{1812.26} & \cellcolor{gray!15}{1817.06} & \cellcolor{gray!15}{1853.56}\\
ETS(M,N,M) & 1827.18 & 1831.98 & 1868.48\\*
\end{longtable*}

OS MODELOS SAO OS MESMO, PODEMO SELECIONAR O SEGUNDO MELHOR

\hypertarget{resuxedduos-3}{%
\subsubsection{Resíduos}\label{resuxedduos-3}}

\hypertarget{estudo-de-desempenho-preditivo}{%
\section{Estudo de desempenho
preditivo}\label{estudo-de-desempenho-preditivo}}

\hypertarget{resultados-da-janela-deslizante}{%
\subsection{Resultados da Janela
Deslizante}\label{resultados-da-janela-deslizante}}

\hypertarget{performance-em-relauxe7uxe3o-aos-horizontes-de-previsuxe3o}{%
\subsection{Performance em relação aos horizontes de
previsão}\label{performance-em-relauxe7uxe3o-aos-horizontes-de-previsuxe3o}}

\hypertarget{arima}{%
\subsubsection{ARIMA}\label{arima}}

\hypertarget{ets}{%
\subsubsection{ETS}\label{ets}}

\hypertarget{resultados}{%
\section{Resultados}\label{resultados}}

apresente em tabelas e gráficos as previsões dos 4 modelos selecionados
e também apresente em uma tabela os resultados de acurácia dos 4 modelos
selecionados e dos modelos benchmarks. Comente os resultados de modo
objetivo;

\hypertarget{apuxeandice}{%
\section{Apêndice}\label{apuxeandice}}

Todo o projeto de composição deste documento pode ser encontrado aqui:
https://github.com/cesar-galvao/trabalhos\_series



\end{document}
