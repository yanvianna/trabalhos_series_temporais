% Options for packages loaded elsewhere
\PassOptionsToPackage{unicode}{hyperref}
\PassOptionsToPackage{hyphens}{url}
\PassOptionsToPackage{dvipsnames,svgnames,x11names}{xcolor}
%
\documentclass[
  letterpaper,
  DIV=11,
  numbers=noendperiod]{scrartcl}

\usepackage{amsmath,amssymb}
\usepackage{iftex}
\ifPDFTeX
  \usepackage[T1]{fontenc}
  \usepackage[utf8]{inputenc}
  \usepackage{textcomp} % provide euro and other symbols
\else % if luatex or xetex
  \usepackage{unicode-math}
  \defaultfontfeatures{Scale=MatchLowercase}
  \defaultfontfeatures[\rmfamily]{Ligatures=TeX,Scale=1}
\fi
\usepackage{lmodern}
\ifPDFTeX\else  
    % xetex/luatex font selection
\fi
% Use upquote if available, for straight quotes in verbatim environments
\IfFileExists{upquote.sty}{\usepackage{upquote}}{}
\IfFileExists{microtype.sty}{% use microtype if available
  \usepackage[]{microtype}
  \UseMicrotypeSet[protrusion]{basicmath} % disable protrusion for tt fonts
}{}
\makeatletter
\@ifundefined{KOMAClassName}{% if non-KOMA class
  \IfFileExists{parskip.sty}{%
    \usepackage{parskip}
  }{% else
    \setlength{\parindent}{0pt}
    \setlength{\parskip}{6pt plus 2pt minus 1pt}}
}{% if KOMA class
  \KOMAoptions{parskip=half}}
\makeatother
\usepackage{xcolor}
\setlength{\emergencystretch}{3em} % prevent overfull lines
\setcounter{secnumdepth}{-\maxdimen} % remove section numbering
% Make \paragraph and \subparagraph free-standing
\ifx\paragraph\undefined\else
  \let\oldparagraph\paragraph
  \renewcommand{\paragraph}[1]{\oldparagraph{#1}\mbox{}}
\fi
\ifx\subparagraph\undefined\else
  \let\oldsubparagraph\subparagraph
  \renewcommand{\subparagraph}[1]{\oldsubparagraph{#1}\mbox{}}
\fi

\usepackage{color}
\usepackage{fancyvrb}
\newcommand{\VerbBar}{|}
\newcommand{\VERB}{\Verb[commandchars=\\\{\}]}
\DefineVerbatimEnvironment{Highlighting}{Verbatim}{commandchars=\\\{\}}
% Add ',fontsize=\small' for more characters per line
\usepackage{framed}
\definecolor{shadecolor}{RGB}{241,243,245}
\newenvironment{Shaded}{\begin{snugshade}}{\end{snugshade}}
\newcommand{\AlertTok}[1]{\textcolor[rgb]{0.68,0.00,0.00}{#1}}
\newcommand{\AnnotationTok}[1]{\textcolor[rgb]{0.37,0.37,0.37}{#1}}
\newcommand{\AttributeTok}[1]{\textcolor[rgb]{0.40,0.45,0.13}{#1}}
\newcommand{\BaseNTok}[1]{\textcolor[rgb]{0.68,0.00,0.00}{#1}}
\newcommand{\BuiltInTok}[1]{\textcolor[rgb]{0.00,0.23,0.31}{#1}}
\newcommand{\CharTok}[1]{\textcolor[rgb]{0.13,0.47,0.30}{#1}}
\newcommand{\CommentTok}[1]{\textcolor[rgb]{0.37,0.37,0.37}{#1}}
\newcommand{\CommentVarTok}[1]{\textcolor[rgb]{0.37,0.37,0.37}{\textit{#1}}}
\newcommand{\ConstantTok}[1]{\textcolor[rgb]{0.56,0.35,0.01}{#1}}
\newcommand{\ControlFlowTok}[1]{\textcolor[rgb]{0.00,0.23,0.31}{#1}}
\newcommand{\DataTypeTok}[1]{\textcolor[rgb]{0.68,0.00,0.00}{#1}}
\newcommand{\DecValTok}[1]{\textcolor[rgb]{0.68,0.00,0.00}{#1}}
\newcommand{\DocumentationTok}[1]{\textcolor[rgb]{0.37,0.37,0.37}{\textit{#1}}}
\newcommand{\ErrorTok}[1]{\textcolor[rgb]{0.68,0.00,0.00}{#1}}
\newcommand{\ExtensionTok}[1]{\textcolor[rgb]{0.00,0.23,0.31}{#1}}
\newcommand{\FloatTok}[1]{\textcolor[rgb]{0.68,0.00,0.00}{#1}}
\newcommand{\FunctionTok}[1]{\textcolor[rgb]{0.28,0.35,0.67}{#1}}
\newcommand{\ImportTok}[1]{\textcolor[rgb]{0.00,0.46,0.62}{#1}}
\newcommand{\InformationTok}[1]{\textcolor[rgb]{0.37,0.37,0.37}{#1}}
\newcommand{\KeywordTok}[1]{\textcolor[rgb]{0.00,0.23,0.31}{#1}}
\newcommand{\NormalTok}[1]{\textcolor[rgb]{0.00,0.23,0.31}{#1}}
\newcommand{\OperatorTok}[1]{\textcolor[rgb]{0.37,0.37,0.37}{#1}}
\newcommand{\OtherTok}[1]{\textcolor[rgb]{0.00,0.23,0.31}{#1}}
\newcommand{\PreprocessorTok}[1]{\textcolor[rgb]{0.68,0.00,0.00}{#1}}
\newcommand{\RegionMarkerTok}[1]{\textcolor[rgb]{0.00,0.23,0.31}{#1}}
\newcommand{\SpecialCharTok}[1]{\textcolor[rgb]{0.37,0.37,0.37}{#1}}
\newcommand{\SpecialStringTok}[1]{\textcolor[rgb]{0.13,0.47,0.30}{#1}}
\newcommand{\StringTok}[1]{\textcolor[rgb]{0.13,0.47,0.30}{#1}}
\newcommand{\VariableTok}[1]{\textcolor[rgb]{0.07,0.07,0.07}{#1}}
\newcommand{\VerbatimStringTok}[1]{\textcolor[rgb]{0.13,0.47,0.30}{#1}}
\newcommand{\WarningTok}[1]{\textcolor[rgb]{0.37,0.37,0.37}{\textit{#1}}}

\providecommand{\tightlist}{%
  \setlength{\itemsep}{0pt}\setlength{\parskip}{0pt}}\usepackage{longtable,booktabs,array}
\usepackage{calc} % for calculating minipage widths
% Correct order of tables after \paragraph or \subparagraph
\usepackage{etoolbox}
\makeatletter
\patchcmd\longtable{\par}{\if@noskipsec\mbox{}\fi\par}{}{}
\makeatother
% Allow footnotes in longtable head/foot
\IfFileExists{footnotehyper.sty}{\usepackage{footnotehyper}}{\usepackage{footnote}}
\makesavenoteenv{longtable}
\usepackage{graphicx}
\makeatletter
\def\maxwidth{\ifdim\Gin@nat@width>\linewidth\linewidth\else\Gin@nat@width\fi}
\def\maxheight{\ifdim\Gin@nat@height>\textheight\textheight\else\Gin@nat@height\fi}
\makeatother
% Scale images if necessary, so that they will not overflow the page
% margins by default, and it is still possible to overwrite the defaults
% using explicit options in \includegraphics[width, height, ...]{}
\setkeys{Gin}{width=\maxwidth,height=\maxheight,keepaspectratio}
% Set default figure placement to htbp
\makeatletter
\def\fps@figure{htbp}
\makeatother

\usepackage[auth-lg]{authblk}
\KOMAoption{captions}{tableheading}
\makeatletter
\makeatother
\makeatletter
\makeatother
\makeatletter
\@ifpackageloaded{caption}{}{\usepackage{caption}}
\AtBeginDocument{%
\ifdefined\contentsname
  \renewcommand*\contentsname{Table of contents}
\else
  \newcommand\contentsname{Table of contents}
\fi
\ifdefined\listfigurename
  \renewcommand*\listfigurename{List of Figures}
\else
  \newcommand\listfigurename{List of Figures}
\fi
\ifdefined\listtablename
  \renewcommand*\listtablename{List of Tables}
\else
  \newcommand\listtablename{List of Tables}
\fi
\ifdefined\figurename
  \renewcommand*\figurename{Figure}
\else
  \newcommand\figurename{Figure}
\fi
\ifdefined\tablename
  \renewcommand*\tablename{Table}
\else
  \newcommand\tablename{Table}
\fi
}
\@ifpackageloaded{float}{}{\usepackage{float}}
\floatstyle{ruled}
\@ifundefined{c@chapter}{\newfloat{codelisting}{h}{lop}}{\newfloat{codelisting}{h}{lop}[chapter]}
\floatname{codelisting}{Listing}
\newcommand*\listoflistings{\listof{codelisting}{List of Listings}}
\makeatother
\makeatletter
\@ifpackageloaded{caption}{}{\usepackage{caption}}
\@ifpackageloaded{subcaption}{}{\usepackage{subcaption}}
\makeatother
\makeatletter
\@ifpackageloaded{tcolorbox}{}{\usepackage[skins,breakable]{tcolorbox}}
\makeatother
\makeatletter
\@ifundefined{shadecolor}{\definecolor{shadecolor}{rgb}{.97, .97, .97}}
\makeatother
\makeatletter
\makeatother
\makeatletter
\makeatother
\ifLuaTeX
  \usepackage{selnolig}  % disable illegal ligatures
\fi
\IfFileExists{bookmark.sty}{\usepackage{bookmark}}{\usepackage{hyperref}}
\IfFileExists{xurl.sty}{\usepackage{xurl}}{} % add URL line breaks if available
\urlstyle{same} % disable monospaced font for URLs
\hypersetup{
  pdftitle={Trabalho Prático 2},
  pdfauthor={Ana Carolina Vianna - 18/0097261; César Augusto Galvão - 19/0011572; Yan Flávio Vianna - 14/0166149},
  colorlinks=true,
  linkcolor={blue},
  filecolor={Maroon},
  citecolor={Blue},
  urlcolor={Blue},
  pdfcreator={LaTeX via pandoc}}

\title{Trabalho Prático 2}
\usepackage{etoolbox}
\makeatletter
\providecommand{\subtitle}[1]{% add subtitle to \maketitle
  \apptocmd{\@title}{\par {\large #1 \par}}{}{}
}
\makeatother
\subtitle{Análise de Séries Temporais - 1/2023}
\author{Ana Carolina Vianna - 18/0097261 \and César Augusto Galvão -
19/0011572 \and Yan Flávio Vianna - 14/0166149}
\date{}

\begin{document}
\maketitle
\ifdefined\Shaded\renewenvironment{Shaded}{\begin{tcolorbox}[enhanced, interior hidden, boxrule=0pt, breakable, sharp corners, frame hidden, borderline west={3pt}{0pt}{shadecolor}]}{\end{tcolorbox}}\fi

\renewcommand*\contentsname{Table of contents}
{
\hypersetup{linkcolor=}
\setcounter{tocdepth}{3}
\tableofcontents
}
\newpage{}

\hypertarget{introduuxe7uxe3o-suxe9rie-selecionada-caracteruxedsticas-e-decomposiuxe7uxe3o}{%
\section{Introdução: série selecionada, características e
decomposição}\label{introduuxe7uxe3o-suxe9rie-selecionada-caracteruxedsticas-e-decomposiuxe7uxe3o}}

A série temporal escolhida foi a de número \emph{id} correspondente a
2183. De acordo com a definição do próprio pacote, refere-se a
\emph{Fluid power shipments - hydraulic index}. Foram realizadas medidas
mensais de 1983 a 1992 e o horizonte de previsão requerido é das 18
ocorrências seguintes.

O gráfico da série, com \emph{in} e \emph{out-sample}, é exposto a
seguir.

\includegraphics{T2_grupo5_files/figure-pdf/plot-serie-total-1.pdf}

A série aparenta ter dois períodos, pelo menos: um ciclo anual e outro
que compreende um período maior. No entanto, ao se tentar decompor a
série com múltiplas sazonalidades, obté-se o seguinte:

\begin{itemize}
\tightlist
\item
  \textbf{Adicionando uma componente sazonal com ciclo menor que 1 ano}
  -- uma das componentes sazonais apresenta heteroscedasticidade;
\item
  \textbf{Adicionando uma componente sazonal com ciclo maior que 1 ano}
  -- resíduos apresentam periodicidade ou heteroscedasticidade.
\end{itemize}

Optou-se portanto pela decomposição STL apenas com a sazonalidade anual,
mas fica evidente que esta decomposição não é adequada quando se avalia
a componente de tendência. A decomposição é exposta a seguir.

\includegraphics{T2_grupo5_files/figure-pdf/decomposicao-mstl-1.pdf}

\hypertarget{modelos-arima-seleuxe7uxe3o-transformauxe7uxf5es-e-resuxedduos}{%
\section{Modelos ARIMA: seleção, transformações e
resíduos}\label{modelos-arima-seleuxe7uxe3o-transformauxe7uxf5es-e-resuxedduos}}

\hypertarget{modelos-ets-seleuxe7uxe3o-transformauxe7uxf5es-e-resuxedduos}{%
\section{Modelos ETS: seleção, transformações e
resíduos}\label{modelos-ets-seleuxe7uxe3o-transformauxe7uxf5es-e-resuxedduos}}

\hypertarget{estudo-de-desempenho-preditivo}{%
\section{Estudo de desempenho
preditivo}\label{estudo-de-desempenho-preditivo}}

\hypertarget{resultados-da-janela-deslizante}{%
\subsection{Resultados da Janela
Deslizante}\label{resultados-da-janela-deslizante}}

\hypertarget{performance-em-relauxe7uxe3o-aos-horizontes-de-previsuxe3o}{%
\subsection{Performance em relação aos horizontes de
previsão}\label{performance-em-relauxe7uxe3o-aos-horizontes-de-previsuxe3o}}

\hypertarget{arima}{%
\subsubsection{ARIMA}\label{arima}}

\hypertarget{ets}{%
\subsubsection{ETS}\label{ets}}

\hypertarget{resultados}{%
\section{Resultados}\label{resultados}}

apresente em tabelas e gráficos as previsões dos 4 modelos selecionados
e também apresente em uma tabela os resultados de acurácia dos 4 modelos
selecionados e dos modelos benchmarks. Comente os resultados de modo
objetivo;

\hypertarget{apuxeandice}{%
\section{Apêndice}\label{apuxeandice}}

Todo o projeto de composição deste documento pode ser encontrado aqui:
https://github.com/cesar-galvao/trabalhos\_series

\begin{Shaded}
\begin{Highlighting}[]
\NormalTok{pacman}\SpecialCharTok{::}\FunctionTok{p\_load}\NormalTok{(Mcomp, tidyverse, forecast, fpp2, xts, tseries, tidymodels)}

\FunctionTok{data}\NormalTok{(M3) }\CommentTok{\#carrega os dados}
\NormalTok{id }\OtherTok{\textless{}{-}} \DecValTok{2546} \CommentTok{\#série temporal escolhida}

\NormalTok{serie }\OtherTok{\textless{}{-}}\NormalTok{ M3[[id]]}

\NormalTok{dados }\OtherTok{\textless{}{-}}\NormalTok{ serie}\SpecialCharTok{$}\NormalTok{x}

\FunctionTok{plot}\NormalTok{(serie, }\AttributeTok{main =} \StringTok{"Série Temporal M3{-}2546"}\NormalTok{)}

\NormalTok{serie}\SpecialCharTok{$}\NormalTok{x }\SpecialCharTok{\%\textgreater{}\%} 
  \FunctionTok{stl}\NormalTok{(}\AttributeTok{s.window =} \DecValTok{7}\NormalTok{, }\AttributeTok{t.window =} \DecValTok{7}\NormalTok{) }\SpecialCharTok{\%\textgreater{}\%}
  \FunctionTok{plot}\NormalTok{(}\AttributeTok{main =} \StringTok{"Decomposição STL (LOESS)"}\NormalTok{)}

\CommentTok{\#sazonalidade diaria e semanal}
\FunctionTok{msts}\NormalTok{(serie}\SpecialCharTok{$}\NormalTok{x, }\AttributeTok{seasonal.periods =} \FunctionTok{c}\NormalTok{(}\DecValTok{12}\NormalTok{, }\DecValTok{6}\NormalTok{))}\SpecialCharTok{\%\textgreater{}\%}
\CommentTok{\#janela de ajuste \textquotesingle{}Cleveland et al (1990)\textquotesingle{}}
  \FunctionTok{mstl}\NormalTok{(., }\AttributeTok{s.window =} \DecValTok{7}\NormalTok{, }\AttributeTok{t.window =} \DecValTok{7}\NormalTok{) }\SpecialCharTok{\%\textgreater{}\%} \FunctionTok{plot}\NormalTok{(}\AttributeTok{main =} \StringTok{"Decomposição MSTL"}\NormalTok{)}

\NormalTok{d }\OtherTok{\textless{}{-}} \FunctionTok{ndiffs}\NormalTok{(serie}\SpecialCharTok{$}\NormalTok{x)}

\NormalTok{D }\OtherTok{\textless{}{-}}\NormalTok{ serie}\SpecialCharTok{$}\NormalTok{x }\SpecialCharTok{\%\textgreater{}\%} \FunctionTok{diff}\NormalTok{() }\SpecialCharTok{\%\textgreater{}\%} \FunctionTok{nsdiffs}\NormalTok{()}

\CommentTok{\#aplica 2 diferenciacoes}
\NormalTok{serie\_d }\OtherTok{\textless{}{-}}\NormalTok{ serie}\SpecialCharTok{$}\NormalTok{x }\SpecialCharTok{\%\textgreater{}\%} \FunctionTok{diff}\NormalTok{(}\AttributeTok{differences =} \DecValTok{2}\NormalTok{)}

\FunctionTok{kpss.test}\NormalTok{(serie\_d)}

\FunctionTok{par}\NormalTok{(}\AttributeTok{mfrow =} \FunctionTok{c}\NormalTok{(}\DecValTok{1}\NormalTok{, }\DecValTok{3}\NormalTok{))}
\FunctionTok{plot}\NormalTok{(serie\_d)}
\FunctionTok{acf}\NormalTok{(serie\_d, }\AttributeTok{lag.max =} \DecValTok{12}\SpecialCharTok{*}\DecValTok{7}\NormalTok{)}
\FunctionTok{pacf}\NormalTok{(serie\_d, }\AttributeTok{lag.max =} \DecValTok{12}\SpecialCharTok{*}\DecValTok{7}\NormalTok{)}

\NormalTok{parcim }\OtherTok{\textless{}{-}} \FunctionTok{character}\NormalTok{()}

\NormalTok{melhor\_AICc }\OtherTok{\textless{}{-}} \ConstantTok{Inf}
\ControlFlowTok{for}\NormalTok{(p }\ControlFlowTok{in} \DecValTok{0}\SpecialCharTok{:}\DecValTok{2}\NormalTok{)\{}
  \ControlFlowTok{for}\NormalTok{(q }\ControlFlowTok{in} \DecValTok{0}\SpecialCharTok{:}\DecValTok{2}\NormalTok{)\{}
    \ControlFlowTok{for}\NormalTok{(P }\ControlFlowTok{in} \DecValTok{0}\SpecialCharTok{:}\DecValTok{1}\NormalTok{)\{}
      \ControlFlowTok{for}\NormalTok{(Q }\ControlFlowTok{in} \DecValTok{0}\SpecialCharTok{:}\DecValTok{1}\NormalTok{)\{}
\NormalTok{        fit }\OtherTok{\textless{}{-}} \FunctionTok{Arima}\NormalTok{(serie}\SpecialCharTok{$}\NormalTok{x, }\AttributeTok{order =} \FunctionTok{c}\NormalTok{(p,}\DecValTok{2}\NormalTok{,q), }\AttributeTok{seasonal =} \FunctionTok{c}\NormalTok{(P, }\DecValTok{0}\NormalTok{, Q))}
        \ControlFlowTok{if}\NormalTok{(fit}\SpecialCharTok{$}\NormalTok{aicc }\SpecialCharTok{\textless{}}\NormalTok{ melhor\_AICc)\{}
\NormalTok{          melhor\_AICc }\OtherTok{\textless{}{-}}\NormalTok{ fit}\SpecialCharTok{$}\NormalTok{aicc}
\NormalTok{          parcim }\OtherTok{\textless{}{-}} \FunctionTok{c}\NormalTok{(}\FunctionTok{paste0}\NormalTok{(}\StringTok{"p = "}\NormalTok{,p,}\StringTok{", q = "}\NormalTok{, q, }\StringTok{", P = "}\NormalTok{, P, }\StringTok{", Q = "}\NormalTok{, Q,}
                             \StringTok{", AICc = "}\NormalTok{, }\FunctionTok{round}\NormalTok{(fit}\SpecialCharTok{$}\NormalTok{aicc,}\DecValTok{4}\NormalTok{)),parcim)}
\NormalTok{        \}}
\NormalTok{      \}}
\NormalTok{    \}}
\NormalTok{  \}}
\NormalTok{\}}



\NormalTok{parcim}


\NormalTok{fit }\OtherTok{\textless{}{-}} \FunctionTok{arima}\NormalTok{(serie}\SpecialCharTok{$}\NormalTok{x, }\AttributeTok{order =} \FunctionTok{c}\NormalTok{(}\DecValTok{0}\NormalTok{,}\DecValTok{2}\NormalTok{,}\DecValTok{1}\NormalTok{), }\AttributeTok{seasonal =} \FunctionTok{c}\NormalTok{(}\DecValTok{1}\NormalTok{,}\DecValTok{0}\NormalTok{,}\DecValTok{0}\NormalTok{), }\AttributeTok{method =} \StringTok{"CSS"}\NormalTok{)}

\NormalTok{fit}\SpecialCharTok{$}\NormalTok{coef }\SpecialCharTok{\%\textgreater{}\%} \FunctionTok{tidy}\NormalTok{()}


\FunctionTok{par}\NormalTok{(}\AttributeTok{mfrow=}\FunctionTok{c}\NormalTok{(}\DecValTok{1}\NormalTok{,}\DecValTok{2}\NormalTok{))}
\NormalTok{E }\OtherTok{\textless{}{-}}\NormalTok{ fit}\SpecialCharTok{$}\NormalTok{residuals }\SpecialCharTok{\%\textgreater{}\%} \FunctionTok{window}\NormalTok{(}\AttributeTok{start=}\FunctionTok{c}\NormalTok{(}\DecValTok{1984}\NormalTok{,}\DecValTok{3}\NormalTok{))}
\NormalTok{E1 }\OtherTok{\textless{}{-}}\NormalTok{ fit}\SpecialCharTok{$}\NormalTok{residuals}
\FunctionTok{plot}\NormalTok{(E1)}
\FunctionTok{plot}\NormalTok{(E, }\AttributeTok{main =} \StringTok{"Processo dos resíduos"}\NormalTok{)}

\FunctionTok{par}\NormalTok{(}\AttributeTok{mfrow=}\FunctionTok{c}\NormalTok{(}\DecValTok{1}\NormalTok{,}\DecValTok{3}\NormalTok{))}
\FunctionTok{plot}\NormalTok{(E)}
\FunctionTok{qqnorm}\NormalTok{(E)}
\FunctionTok{qqline}\NormalTok{(E)}
\FunctionTok{acf}\NormalTok{(E, }\AttributeTok{lag.max=}\DecValTok{12}\SpecialCharTok{*}\DecValTok{5}\NormalTok{)}


\FunctionTok{bind\_rows}\NormalTok{(}
\NormalTok{  tseries}\SpecialCharTok{::}\FunctionTok{kpss.test}\NormalTok{(E) }\SpecialCharTok{\%\textgreater{}\%} \FunctionTok{tidy}\NormalTok{() }\SpecialCharTok{\%\textgreater{}\%} \FunctionTok{select}\NormalTok{(}\SpecialCharTok{{-}}\NormalTok{parameter),}
  \FunctionTok{Box.test}\NormalTok{(E, }\AttributeTok{lag =} \DecValTok{20}\NormalTok{, }\AttributeTok{type =} \StringTok{"Ljung{-}Box"}\NormalTok{)}\SpecialCharTok{\%\textgreater{}\%} \FunctionTok{tidy}\NormalTok{() }\SpecialCharTok{\%\textgreater{}\%} \FunctionTok{select}\NormalTok{(}\SpecialCharTok{{-}}\NormalTok{parameter)}
  
  \FunctionTok{auto.arima}\NormalTok{(serie}\SpecialCharTok{$}\NormalTok{x, }\AttributeTok{d =} \DecValTok{2}\NormalTok{, }\AttributeTok{D =} \DecValTok{0}\NormalTok{)}
\end{Highlighting}
\end{Shaded}




\end{document}
